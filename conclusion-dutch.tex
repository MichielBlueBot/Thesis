\chapter{Conclusion}
\label{cha:conclusion}
Het eerste hoofdstuk toont aan dat er een nood is aan data-integratie methoden door de steeds groeiende datasets waarmee we moeten werken. Door op een slimme manier verschillende datasets te integreren kunnen we predictieve modellen bouwen die beter presteren. \\ \\
Vervolgens beschreven we wat predictieve modellen inhouden en we toonden twee concrete voorbeelden: logistieke regressie - en cox modellen.  \\ \\
Met deze informatie konden we vervolgens de verschillende integratie strategi\"en voorstellen. De eerste strategie, vroege integratie, voegt alle datasets aan elkaar door overeenkomende datapunten te linken. De tweede strategie, late integratie, bouwt individuele modellen voor elke dataset, ge\"inspireerd op het concept van ensemble-leren. De laatste strategie, parti\"ele integratie, bouwt het predictief model in twee stappen en maakt gebruik van de variabele selectie eigenschap in de lasso regularisatie. \\ \\
Om aan te tonen dat de verschillende integratie strategie\"en een impact hebben op de performantie van de predictieve modellen, passen we de strategie\"en toe in twee case studies met echte gegevens over kankerpati\"enten. De eerste studie gebruikt logistieke regressie modellen. De data voor deze studie komt van het Universitair Ziekenhuis te Leuven en betreft scan-gegevens voor pati\"enten met darmkanker. De resultaten van deze studie tonen een duidelijke verbetering van de performantie door het gebruik van integratie. De tweede case studie gebruikt cox proportionele risico modellen. De data van deze studie kwam van de online TCGA database. De resultaten in deze studie waren minder evindent, maar we kunnen toch besluiten dat ook in dit geval de ge\"integreerde modellen beter presteren dan de individuele modellen. En specifiek het partieel ge\"integreerde model bleek het beste te zijn. Beide studies leveren daarom evidentie aan dat integratie van datasets inderdaad de performantie ten goede komt, en dat verschillende strategie\"en een verschillende impact hebben. \\ \\
Het moet echter duidelijk zijn dat dit niet het einde van de studie is. In dit thesis toonden we drie verschillende integratie strategie\"en, maar er zijn duidelijk nog veel andere manier mogelijk om datasets te integreren. Verder spitste de case studies in dit geval zich toe op twee lineaire modellen: logistieke regressie- en cox modellen. Een interessante uitbreiding is om te kijken of deze claims ook gelden voor andere (niet-lineaire) methoden. \\ \\
Daarnaast moeten we in ons achterhoofd onthouden dat deze studie kadert in een grotere studie wiens doel het is om patronen en nieuwe inzichten te vinden in de grote hoeveelheden kankergegevens waarover we beschikken. Door betere predictieve modellen te bouwen kunnen we in de toekomst nieuwe inzichten verkrijgen in de manier waarop kankers ontstaan en zich ontwikkelen. Dit zal op zijn beurt in de toekomst een weg bieden naar nieuwe behandelingen voor kanker.

%%% Local Variables: 
%%% mode: latex
%%% TeX-master: "thesis"
%%% End: 
