\chapter{Machine Learning Models}
\label{cha:1}
In this chapter the different machine learning methods that are used are being explained in detail. This is necessary because later on I will explain different integration strategies, and these often depend on the specific structure of a machine learning method. The first section explains what Generalized Linear Models are. This name actually indicates a whole collection of different methods, but there is one method that will be of particular interest to us, called 'logistic regression'. This section will be the most extensive as it explains a lot of concepts that can be reused later on, and also because this method is the one that I have used the most in my thesis. \\
The next section will explain the Cox Proportional Hazards method. This method is used when the data we are dealing with is so-called survival data. \\
Lastly I will briefly mention Neural Networks and why they are important.

\section{Generalized Linear Models}
When we think of classical regression, you can imagine a set of numeric explanatory or input variables and a numerical dependent or output variable. In this case we assume that when we try to estimate the output variable by using the input variables, the error that we make will be normally distributed.

\lipsum[55]

\subsection{An item}
Please don't abuse enumerations: short enumerations shouldn't use
``\verb|itemize|'' or ``\texttt{enumerate}'' environments.
So \emph{never write}: 
\begin{quote}
  The Eiffel tower has three floors:
  \begin{itemize}
  \item the first one;
  \item the second one;
  \item the third one.
  \end{itemize}
\end{quote}
But write:
\begin{quote}
  The Eiffel tower has three floors: the first one, the second one, and the
  third one.
\end{quote}

\section{A Second Topic}
\lipsum[64]

\subsection{Another item}
\lipsum[56-57]

\section{Conclusion}
The final section of the chapter gives an overview of the important results
of this chapter. This implies that the introductory chapter and the
concluding chapter don't need a conclusion.

\lipsum[66]

%%% Local Variables: 
%%% mode: latex
%%% TeX-master: "thesis"
%%% End: 
