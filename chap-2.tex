\chapter{Cox proportional hazards models}
\label{cha:cox}

\section{Introduction}
\label{sec:cox-introduction}
In this chapter I will explain the cox proportional hazards model. These are models that are used in survival analysis. The first section will explain what survival analysis means and how it is different from the generalized linear models. Next I will define what the proportional hazards assumption is and lastly I will show how these models can be used in practice.

\section{Survival analysis}
\label{sec:cox-survival-analysis}
Survival analysis points to the fact that the outcome variables are timed events. A common example in biomedical research would be the time between the start of a patients treatment and the time of death. Note however that not all patients have to die in order to be useful in a survival analysis. A second outcome variable is used to indicate whether the patient survived until the end of the study, or not. This binary variable is called the censoring variable. \\ \\
A typical outcome in survival analysis is thus represented by two vectors: a time to event vector and a censoring vector. An example is shown in table \ref{tab:cox-example-outcome}.
\begin{table}
	\centering
	\begin{tabular}{cc}
		\toprule
		Time (days) & Censor\\
		\midrule
		249 & 1 \\
		345 & 1 \\
		152 & 1 \\
		452 & 0 \\
		120 & 1 \\
		... & ... \\
		\bottomrule
	\end{tabular}
	\caption{Example outcomes for survival data}
	\label{tab:cox-example-outcome}
\end{table}
The idea of survival analysis is now to find a pattern between a set of explanatory variables and the time-to-event. A common term used in survival analysis is hazard, meaning 'risk of the event occurring'. A resulting model from survival analysis usually contains two components. The first component is a $\lambda_{0}(t)$ baseline hazard function. This function describes how the risk of the event occurring changes with time, assuming there is no influence from any of the explanatory variables. The second component describes the influence of each explanatory variable on the hazard. This brings us to the notion of proportional hazards.

\section{Cox proportional hazards model}
\label{sec:cox-proportional-hazards-model}
\subsection{Proportional hazards condition}
The proportional hazards condition means that we assume that hazard ratios are independent of time. A hazard ratio is the relative risk between two entities (patients). Furthermore the condition states that changes in the explanatory variables have an exponential effect on the hazard. \\ \\
If we assume the proportional hazards condition to be true then this gives us the opportunity to estimate the effect of explanatory variables without dealing with the underlying baseline hazard function. This was an observation made by Sir David Cox //TODO REFERENCE COX 1972, hence the name of the model. We can represent the cox model as follows:
$$
\lambda_{i}(t) = \lambda_{0}(t)e^{X_{i1}\beta_{1} + ... + X_{iN}\beta_{N}}
$$
where
\begin{itemize}
	\item $\lambda_{i}(t)$ is the hazard for patient i at time t.
	\item $\lambda_{0}(t)$ is the baseline hazard at time t.
	\item $X_{i1} ... X_{iN}$ are the values of the explanatory variables for patient i.
	\item $beta_{1} ... beta_{N}$ are the values of the co\"effici\"ents for each explanatory variable.
\end{itemize}
Now, let's look at relative risks between two patients:
$$
\frac{\lambda_{i}(t)}{\lambda_{j}(t)} 
= \frac{\lambda_{0}(t)e^{X_{i1}\beta_{1} + ... + X_{iN}\beta_{N}}}{\lambda_{0}(t)e^{X_{j1}\beta_{1} + ... + X_{jN}\beta_{N}}}
= e^{(X_{i1}-X_{j1})\beta_{1} + ... + (X_{iN}-X_{jN})\beta_{N}}
$$
This quantity is called the hazard ratio. Notice that this ratio does not depend on the time. This means that if at the start of the study a patient has twice the risk of the event occurring compared to another patient, it will have twice the risk at any other time aswell. It's risk remains proportional, independent of time. \\ \\
We can also compute the effect on the hazard for a unit increase in an explanatory variable $X_{j}$. The subscripts i to indicate the patients have been omitted since this is independent of a specific patient. The hazard ratio then becomes:
$$
\frac{\lambda(t|X_{j}+1)}{\lambda(t|X_{j})} = e^{(X_{j}+1-X_{j})\beta_{j}} = e^{\beta_{j}}
$$
It is directly given by the exponential of the co\"effici\"ent for the explanatory variable. This reflects the proportional hazards assumption that hazard ratios do not depend on time.
\subsection{Using the hazard ratio}
The proportional hazards assumption allows us to compute relative risks or hazard ratios without knowing the actual baseline hazard function. Knowing just relative ratios can be very useful however. Imagine the following scenario: we want to test if a certain drug increases the survivability of patients with a specific disease. We randomly give half of the patients the actual drug, the other half receives a placebo (or nothing at all). Performing a survival analysis using a cox model then allows us to compute the ratio of the hazards for the two groups. The co\"effici\"ent for the explanatory variable that indicates which patients received the drug and which did not then tells us the effect of the drug on the hazard. We could then make a statement such as: "Taking the drug tends to half the risk of the event occurring (at any time)." This is obviously a useful result to support further research or development for the drug.

\subsection{Kaplan-Meier survival curves}


\section{Conclusion}
\label{sec:cox-conclusion}
In this chapter I have explained the cox proportional hazards method. I have shown that it is used for survival analysis, where the targets are time and censoring variables. I have explained the proportional hazards assumption and shown its implications. Lastly I have shown how we can compute survival curves using the Kaplan-Meier method.

%%% Local Variables: 
%%% mode: latex
%%% TeX-master: "thesis"
%%% End: 
