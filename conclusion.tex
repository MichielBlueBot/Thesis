\chapter{Conclusion}
\label{cha:conclusion}
The first chapter in this thesis shows that due to increasingly larger datasets there is a need for data integration methods that can join together datasets from different sources to create more powerful predictive models than the ones we can build from individual datasets. The aim of this thesis is to show that indeed different integration strategies have a varying impact on the performance of the resulting predictive models. \\ \\
Chapters two and three respectively give an extensive overview of two machine learning methods: logistic regression- (chapter \ref{cha:glm}) and cox proportional hazard analysis (chapter \ref{cha:cox}). By providing a solid foundation for the concepts of training, overfitting, regularization, validation and survival analysis, the reader should be able to fully grasp the integration strategies presented in the next chapter. \\ \\
The fourth chapter explains the three integration strategies that we developed. The first is called early integration which is based on concatenation of datasets. The second is called late integration which builds on the ensemble learning concept. And the third is called intermediate integration which makes extensive use of the variable selection present in the lasso regularization. \\ \\
In order to show that these different integration strategies have a varying impact on the performance, we applied these strategies in two case studies on real cancer data (chapter \ref{cha:evaluation}). The first study used logistic regression to build the predictive models. The data for this study came from the University Hospital of Leuven and concerned imaging data for patients with colorectal cancer. The variable that we tried to predict in this case was a binarized version of the pathologic stage (see appendix \ref{app:cancer-staging}). The results of the study showed an improvement in performance when using integration strategies, especially the intermediate integration strategy. The second study used cox proportional hazards analysis to build the predictive models. The data for this study came from TCGA which is a publicly available database for cancer data. More specifically we used the Copy Number Variation (CNV), messenger RNA and micro RNA datasets for colon cancer. The results in this study were a bit less evident due to insignificance of many of the resulting models. However we can still conclude that also in this study the intermediate integration strategy proved to provide the best models. Both studies therefore provided evidence to support the aim of the thesis that integration strategies have an effect on the performance of predictive models. \\ \\
In order to facilitate the computation and evaluation of the predictive models in this thesis we have developed an interactive application in the programming language R. The application allows us to easily train and evaluate the predictive models presented in this thesis. The application is built keeping in mind that this thesis is only part of a greater study. We have showed in two concrete studies that the integration methods have an impact, but this only creates more possibilities for future research. We have evaluated three integration strategies, but it should be evident that there are many more ways of integrating data and all of these should be explored in the future. Furthermore, we have used logistic regression- and cox proportional hazard models to support our aim, but there are many other machine learning methods for which similar research needs to be done. \\ \\
And with that in mind we have to remember that this study is encompassed by an even larger study aim which is to find patterns in the large amounts of cancer data that is being produced. By building better predictive models, we can develop novel insights into the way cancer originates and develops. Paving the road for better prognosis and treatment of cancer in the future.

%%% Local Variables: 
%%% mode: latex
%%% TeX-master: "thesis"
%%% End: 
